\section{Introduction}
%\writer{Thibaud}



\begin{quotation}
Petri nets, as graphical and mathematical tools, provide a uniform environment for modelling, formal analysis, and design of discrete event systems. \footnote{Petri Nets and Industrial Applications: A Tutorial. Richard Zurawski, MengChu Zhou.\url{Petri Nets and Industrial Applications: A Tutorial. Richard Zurawski, MengChu Zhou. (http://www.cs.uga.edu/~eileen/WebEffectiveness/Papers/Petri netsAndIndustrialApplications.pdf)}}
\end{quotation}
%[^1]: Petri Nets and Industrial Applications: A Tutorial. Richard Zurawski, MengChu Zhou. (http://www.cs.uga.edu/~eileen/WebEffectiveness/Papers/Petri netsAndIndustrialApplications.pdf)

Petri nets are used as a means to model systems, but, as they are a mathematical concept, they are not always easy to understand for the ordinary user. Complex systems can be modelled with Petri nets, and usually, this would be an engineer's job. Even though engineers can easily create and understand their own Petri nets, every team member in a company would like to be able to understand what a Petri net is about without having any knowledge of the concepts behind them.

Therefore, the project aims at creating a 3D visualisation from a Petri net, to allow non-Petri net experts to actually get a feel of how the model works to understand and validate a system.

However, this 3D visualisation that we would like to present comes with a problem: Petri nets were not intended to have a 3D representation. For instance, there is no graphical concept or way to say that a particular Petri net would look like a train track because it was used to model a railway system. 

Thus, our goal for this project is the following: Providing an extension to Petri net models to make their 3D visualisation possible. For this purpose, we have imagined a simple link between a Petri net and a 3D visualisation.

