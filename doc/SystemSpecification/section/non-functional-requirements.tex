\section{Non functional requirements}
\label{sec:non-functional}
\writer{Mikko}

In addition to implementing the required functionality of the product, a number of non-functional requirements were also taken into account. This sections describes into more detail the latter.

\subsection{Implementation constraints}
The software must be implemented as plug-in to the Eclipse framework and it is based on the following technologies:
\begin{enumerate}
	\item Eclipse Modeling Framework, EMF v. 2.91
	\item Graphical Modeling Framework, GMF 3.91
	\item Graphical Editing Framework, GEF 1.70
	\item ePNK v. 1.0.0
	\item JMonkey v. 3.0
\end{enumerate}

\subsection{Documentation}
Development must be documented and documentations must be delivered according to the following deadlines:
\begin{enumerate}
	\item Project definition, week 39
	\item UML diagrams, week 41
	\item System specification, week 44
	\item Handbook, draft, 47
	\item Test documentation, 51
	\item Final documentation, 51
\end{enumerate}

\subsection{Quality Assurance}
The procedures that ensure the quality of this software product are described in the following section.
\begin{enumerate}
	\item Quality of documentation
	\begin{enumerate}
		\item Each part of the document is realized by multiple persons in order to decrease the amount of typical errors.
		\item Each part of the documentation is reviewed and feedback is given by each member of the team. Depending on the feedback, appropriate measures are taken and the documentation is edited.
		\item Internal deadlines for documentation are set, so that there is enough time allocated for internal reviews before the real deadline.
	\end{enumerate}
	\item Quality of implementation
	\begin{enumerate}
		\item Each implementation is reviewed by at least two persons in order to decrease the amount of typical errors.
		\item Commenting on manually generated code shall use the Javadoc standard.
		\item A test strategy shall be developed to test the implementation. The strategy will consists of different types of testing such as unit test, acceptance test and system testing. Based on the functionality of each component, a suitable type of testing will be assigned to it. 
	\end{enumerate}
\end{enumerate}

\subsection{Performance requirements}
The software can be excepted to fulfill following performance requirements.
\begin{enumerate}
	\item Satisfactory response times for geometry and PetriNet editor related commands are immediate.
	\item Satisfactory response time for simulator related commands should be immediate for small models. No upper boundary for the size of the model is made. Thus response time for larger models will gradually approach infinity depending on the size of the model.
	\item Generally the amount of user request as scale of time is small. When the relative amount of requests is large they are simple, such as writing in editors, and when the relative amount of requests is small they are more advanced, such as play command for 3D simulator.
	\item The number of user per software will be one.
\end{enumerate}