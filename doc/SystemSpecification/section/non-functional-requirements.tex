\section{Non functional requirements}
\label{sec:non-functional}
\writer{Mikko}

In addition to implementing the required functionality of the product, a number of non-functional requirements were also taken into account. This sections describes into more detail the latter.

\subsection{Implementation constraints}
The software must be implemented as plug-in to the Eclipse framework and it is based on the following technologies:
\begin{enumerate}
	\item Eclipse Modeling Framework, EMF v. 2.91
	\item Graphical Modeling Framework, GMF 3.91
	\item Graphical Editing Framework, GEF 1.70
	\item ePNK v. 1.0.0
	\item JMonkey v. 3.0
\end{enumerate}

\subsection{Documentation}
Development must be documented and documentations must be delivered according to the following deadlines:
\begin{enumerate}
	\item Project definition, week 39
	\item UML diagrams, week 41
	\item System specification, week 44
	\item Handbook, draft, week 47
	\item Test documentation, week 51
	\item Final documentation, week 51
\end{enumerate}

\subsection{Quality Assurance}
The procedures that ensure the quality of this software product are described in the following section.
\begin{enumerate}
	\item Quality of documentation
	\begin{enumerate}
		\item Each part of the document is realized by multiple persons in order to decrease the amount of typical errors.
		\item Each part of the documentation is reviewed and feedback is given by each member of the team. Depending on the feedback, appropriate measures are taken and the documentation is edited.
		\item Internal deadlines for documentation are set, so that there is enough time allocated for internal reviews before the real deadline.
	\end{enumerate}
	\item Quality of implementation
	\begin{enumerate}
		\item Each implementation is reviewed by at least two persons in order to decrease the amount of typical errors.
		\item Commenting on manually generated code shall use the Javadoc standard.
		\item A test strategy shall be developed to test the implementation. It will consists of component and functional testing. Component testing will be performed by doing unit tests on manually created methods. Functional testing will be done by performing common user cases and compare the result to the excepted results.
	\end{enumerate}
\end{enumerate}

\subsection{Performance requirements}
The software can be excepted to fulfill following performance requirements.
\begin{enumerate}
	\item Satisfactory response times for geometry and PetriNet editor related commands are immediate.
	\item Satisfactory response time for simulator related commands should be immediate for small models. No upper boundary for the size of the model is made. Thus response time for larger models will gradually approach infinity depending on the size of the model.
	\item Generally the amount of user request as scale of time is small. When the relative amount of requests is large they are simple, such as writing in editors, and when the relative amount of requests is small they are more advanced, such as play command for 3D simulator.
	\item The number of user per software will be one.
\end{enumerate}

\subsection{Usability requirements}
The software can be excepted to fulfill following usability or ease of use requirements.
\begin{enumerate}
	\item Editors and simulator should have familiar interfaces, so that learning to use them is efficient as possible.
	\item The less the user needs to know about software and still be able to use it effectively the less technical oriented the user needs to be. Thus increasing the availability to the software.
	\item The software is divided into multiple parts in order to enable users with different levels of expertise to contribute in working with the software.
	\item User input should tend towards point-and-click style of usage.
\end{enumerate}