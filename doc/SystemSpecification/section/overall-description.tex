\section{Overall description}
%\writer{Thibaud}

This section describes the software and provides a brief explanation on how the system works.

\subsection{What are Petri nets?}

Petri nets are a graphical and mathematical modelling tool for describing concurrent and distributed systems. Some examples of their applications are workflow management, embedded systems or traffic control. The main advantages of Petri nets are their graphical notation, their simplicity on the semantics, and their rich theory for analysing their behaviour. However, using the Petri net graphical notation for understanding a complex system is quite hard, and thus a user-oriented visualization is required in a way that is understandable to users which are not necessarily familiar with Petri nets.


\begin{figure}[htp]
\begin{center}
  \includegraphics[width=0.8\textwidth]{image/petrinet_diagram.png}
  \caption{An example of a basic Petri net}
  \label{fig:petrinet}
\end{center}
\end{figure}

Figure \ref{fig:petrinet} presents an example of a basic Petri net. There are four different elements:

\begin{itemize}
\item Places: graphically represented by circles, represent conditions.
\item Transitions: graphically represented by squares, represent events.
\item Arcs: indicate which places are preconditions/postconditions for which transitions.
\item Tokens: graphically represented by the dots, represent the elements that move in the Petri net along the places through the transitions.
\end{itemize}


Furthermore, we can see the example of Figure \ref{fig:petrinet} as a model for a railway with a traffic light. The token in Place P1 represents a train moving on a railway. When this train arrives to Place P1, the transition T will fire if the conditions are met. The only condition for a transition to be fired is that a token should be present on each of the incoming places of the transition.

In this scenario, the conditions are not met, as the required token in Place Ptl is missing. The visualization of this scenario would be a traffic light with a red light. If Place I had a token on it, it would generate a token that will turn the traffic light to green and thus the train will move. 

%If a user clicks on the Place I, it would generate a token that will turn the traffic light to green and thus the train will move.

With the aim of creating a visualization of the scenarios for the above Petri net model, an application tool is needed in order to allow the user to define where each of the elements are represented in the 3D world (geometry editor) as well as how these elements are represented (shape) and how they behave (animation). 

\subsection{Adding geometry to Petri nets}
The problem this project is tackled with is simple: We need a way to link the Petri net model to a 3D visualisation, this is, to add extra information so that it can be visualized. Once a Petri net model is created and its real life design is well-designed in the user's mind, what we call a "Geometry" and "Appearance" are created.

For this purpose, there is a need of a geometry editor which will be used to assign a two dimensional location to the elements defined in the Petri net, and of an appearance editor to take care of the shape of the elements.

\subsection{Adding appearance to Petri net objects}
\label{sec:appearance}

Once the geometry problem is solved, a shape for each object should be defined. For instance, if places represent tracks, the shape, texture and other attributes should be linked to that object.

For this purpose, there is a need of an appearance editor which will be used to assign 3D visualization features to the elements defined in the Petri net.

\subsection{Configuration}

Before running the simulation, a definition of how the previous models are connected is needed. This is done in the configuration step as well as the validity check for the Petri net's connections to the geometry. 

\subsection{Simulating Petri nets}
The next step in visualizing Petri nets is add descriptive visuals. With the information provided by the Petri net, geometry and appearance as defined in the configuration file, the simulation is set up.

\begin{figure}[htp]
\begin{center}
  \includegraphics[width=0.4\textwidth]{image/example_petrinet.png}
  \caption{A simple extended Petri net}
  \label{fig:example_petrinet}
\end{center}
\end{figure}

\begin{figure}[htp]
\begin{center}
  \includegraphics[width=0.4\textwidth]{image/example_petrinet_geometry.png}
  \caption{A simple geometry}
  \label{fig:example_petrinet_geometry}
\end{center}
\end{figure}

Using 3D models, textures and animations, the Petri net becomes easier to understand for the user. Continuing with the railway example, tokens become trains that move on tracks, which are places. As the tokens move from place to place, the train is animated in the 3D visualization along the tracks.

Figure \ref{fig:3steps}  shows how a simple 3D simulation of a train track and a train is made out of Petri net model and a simple geometry. Place P1 references L1 as its geometry and also has the shape of a track. The token on place P1 has the shape of a train and it moves with an animation defined by the place P1.

\subsection{General product description}

The overall description is a brief introduction to how our software is designed. For this purpose, it contains a description of the actors involved in its use, and also a description of how the software works. \newline

The following persons are all actors of our software:

\begin{itemize}
  \item Petri net engineer: An engineer whose role is to design and develop a Petri net which suits the company needs. \newline
  For instance, this could be an engineer working at a railway company and in charge of modelling the railway system using Petri nets.
  \item End user: An actor to whom the Petri net 3D-Visualization would be presented, or an actor who is in charge of presenting the Petri net to another. \newline
	For instance, this could be a manager wanting to see what a Petri net represents. 
\end{itemize}

The software is built around three different concepts: 

\begin{itemize}
  \item Editors: A component responsible for handling the creation and design of one of our models. \newline
	For example: A Petri net editor to create a Petri net model. \newline
	An editor is presented to the user as a GUI in an Eclipse window.
  \item Simulator: A simulator is, as its name says, a component capable of simulating a Petri net.
	The simulator is handling all the decisions regarding the Petri net and its behaviour. 
	It then communicates the decisions to a 3D Engine responsible for the visualization.
  \item 3D Engine: A 3D Engine is responsible for the visualization of a Petri net in three dimensions.
	It receives input from the Petri net simulator, and also communicates to the simulator when a user performs a certain type of action. \newline
	For example: the user clicks on a specific Place of the Petri net, this triggers a message from the 3D Engine to the Petri net simulator 
\end{itemize}

\subsection{Basic functionality}


This software allows the engineers to create a Petri net 3D Visualization using a combination of the different editors built for this project.
The files created with each of the following editors are to be combined using the configuration editor: \newline
Petri net editor, Geometry editor, Appearance editor. 

Once the files are linked together in the configuration editor, a 3D Visualization can be launched in the workbench. \newline

The simulator then initializes the 3D Engine with all the informations needed for the visualization. The 3D Engine then relies on the simulator to know which next move it should perform on the Petri net. \newline
To interact with the simulation, the user can either click or press certain keyboard buttons in the GUI for the visualization. \newline

Figure 1. shows the different components of our system and their connections.

Figure 2. shows a simple case of how the system could be used.
