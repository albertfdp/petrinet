\section{Glossary}
\begin{description}

\item [GUI] Graphical User Interface

\item [Petri net] A mathematical and graphical model for the description of distributed systems. It is a directed bipartite graph, in which nodes represent transitions and places. The directed arcs describe which places are pre- and/or postconditions for which transitions. [source wiki]

\item [Synchronization] Transition finings are synchronized on the occurrences of external events, such as when animation is finished or user triggers the transition.

\item [Token] Petri Net element which moves along the Petri net places through transitions.

\item [Use case] A list of steps defining interactions between a role (e.g. “Technical user”), also known as an actor, and a system for achieving a goal. The actor can be a human or an external system

\item [3D] Three dimensional.

\item [ePNK] Eclipse Petri Net Kernel.

\item [Bendpoint] Bendpoint is a point, which functions as a control point for parametric curves allowing “bending” of said curve.

\item [Bézier-curve] Bézier-curve is a parametric curve, where shape is defined by blend of different control points.

\item [Catmull-rom] Catmull-rom splines are a family of cubic interpolating splines formulated such that tangent at each point of the spline is calculated using the previous and next point on the spline. (http://www.cs.cmu.edu/~462/projects/assn2/assn2/catmullRom.pdf)

\item [Connector]  Connector serves as either first or the last point of line.

\item [Geometry object] Geometry object can be either point or line.

\item [Input Point] Input Point is a point, where user is allowed to add tokens during the simulation.

\item [Javadoc standard] Format used by Javadoc is the industry standard for documenting Java classes.

\item [Parametric curve] Parametric curve is a mathematical curve defined by point locations.

\item [Point] Point is a coordinate location in a two dimensional plane.

 \end{description}
